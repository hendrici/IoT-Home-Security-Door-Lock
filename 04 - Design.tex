\chapter{Design}
Previous to any finite design decisions, Use Case Diagrams and System Boundaries were created in Week 1. The design of the smart lock involved important decisions about hardware and software alike. These decisions were made in the hardware identification task of our project and were completed by May 17th. The ESP32 was chosen as the microcontroller as it has WiFi capabilities built into the development board, making it compatible with the mobile app portion of the requirements. It also has an operating speed that is compatible with the peripherals chosen for the servo, LCD, and keypad. The keypad chosen was selected because of its durability in comparison to the traditional 12-key membrane keypad. This keypad uses the same logic as keypads used in previous classes so it is compatible with both prior knowledge and the chosen micro-controller. The servo and LCD chosen were also used in previous classes so there was familiarity with their functionality. A fixture was designed and 3D printed to house the peripherals and test their integration of them accordingly. During the Programming Language Identification task and due to the prior knowledge of the group members, the language chosen was C, using driver-level programming. The ultimate design while not finalized, has been laid out for future completion in the last two weeks of the course.